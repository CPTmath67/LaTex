\documentclass[a4paper,10pt,openany]{memoir}
\usepackage[utf8]{inputenc}
%Support de langue française
\usepackage[french]{babel}
%Permet d'inclure des images
\usepackage{graphicx}
%Mise en foem de codes source
\usepackage{listings}
%Nécéssaire pour l'affichage des couleurs
\usepackage{color}
%Liens internes
\usepackage{hyperref}
%Permet de l'utilisation de liens hypertexte 
\usepackage{url}

%Définitions d'options par défaut 
\linespread{1.1}
\lstset{caption={Descriptive Caption Text},language=make,label=DescriptiveLabel,breaklines=true,showtabs=false,commentstyle=\color{red},keywordstyle=\color{blue}, title=\lstname, numbers=left,numberstyle=\tiny}
\renewcommand*{\familydefault}{\sfdefault}
% Title Page
\title{\textbf{Copenhague}}
\author{Mathieu KERN\\
\emph{Contact}: \href{mailto:kernmathieu@gmail.com}{kernmathieu@gmail.com}\\
IUT Charlemagne\\
}




\begin{document}
\maketitle
\begin{abstract}
\large{
\textbf{Copenhague} est la capitale et la plus grande ville du Danemark. Son nom danois, København,
déformation de Købmandshavn, le port des commerçants, rappelle sa position stratégique à l'entrée de la 
mer Baltique. Le nom français est, lui, dérivé de l'allemand Kopenhagen.
\textbf{Copenhague} est la ville , du gouvernement et de la monarchie danoise.
Le premier fort fut bâti dans la ville en 1167. La ville de \textbf{Copenhague} ou, 
plus officiellement, la commune de \textbf{Copenhague} (Københavns Kommune) à proprement parler, 
compte 557 920 habitants (janvier 2008) (voir également les articles sur la ville de \textbf{Copenhague} 
et le Grand \textbf{Copenhague}). Néanmoins, le nom de \textbf{Copenhague} est généralement donné à l'ensemble du 
comté de \textbf{Copenhague}, qui regroupe 1 645 825 d'habitants (janvier 2008).
Son maire est, depuis le 1\ier janvier 2010, le social-démocrate Frank Jensen.
Ses habitants s'appellent, en français, les Copenhaguois(es).}
\end{abstract}
\pagebreak

\tableofcontents

\part{La ville}

\chapter{Histoire et géographie de la ville}

\subsection*{Premieres colonies}
{\footnotesize
Bien que les premiers documents historiques de \textbf{Copenhague} sont datés de la fin du XIIe siècle,
les découvertes archéologiques récentes dans le cadre des travaux sur le métro de la ville ont révélé
les vestiges d'un manoir de grand marchand construit environ vers 1020 ap. J.-C. près de la place 
publique Kongens Nytorv ainsi que les restes d'une ancienne église, avec des tombes datant du XIe 
siècle près du lieu où se réunit la rue commerçante Strøget et la place de l'hôtel de ville Rådhuspladsen2.}

{\large Ces découvertes indiquent que les origines de \emph{\textbf{Copenhague}} en tant que ville remontent au moins au 
11ème siècle. De plus, les découvertes importantes d'outils en silex de la région fournissent des 
preuves des établissements humains datant de l'Âge de la pierre. Plusieurs historiens croient que 
la ville moderne de \textbf{Copenhague} prend racine à la fin de l'Âge des Vikings et que celle-ci a été probablement
fondée par Sven Ier de Danemark.}

{\huge Des fouilles ont mis en évidence l'existence de deux colonies au XIe siècle. La première était située
entre les actuelles rues Mikkel Bryggersgade, Vestergade, Gammeltorv/Nytorv et Løngangsstræde, ce qui 
correspondait à peu près à la ligne côtière de l'époque. La seconde, plus petite correspond à l'actuel 
Kongens Nytorv.

\textsc{
La première mention de la ville se situe en 1043 dans la Knýtlinga saga sous le nom de port (Hafnæ, puis 
Hafnia), où il est dit que Sven II de Danemark s'y réfugie après avoir été battu par Magnus Ier de Norvège.
Dans la Geste des Danois, écrite dans les années 1200, Saxo Grammaticus se réfère à la ville sous le nom de }}
\nopagebreak
\subsection*{Absalon}
\textmd{
En 1157, Valdemar Ier de Danemark fait don de la ville et des villages des environs à Absalon, évêque de
Roskilde3. La lettre de don originale est perdue3 mais la lettre de confirmation du pape Urbain III du 21 
octobre 1186 est conservée4. Absalon y construit alors un château en 1167, ce qui marqua le début de la 
montée en puissance de la ville. Durant les années qui suivirent, la taille de la ville décupla, plusieurs
églises et abbayes furent construites (dont la cathédrale Notre-Dame, détruite aujourd'hui, l'édifice actuel
datant de 1829 ; l'église du Saint-Esprit et l'église Saint-Pierre) et l'économie se développa grâce au 
commerce du hareng.}

\subsection*{Géographie}
\textrm{
La ville de \textbf{Copenhague} est située sur la côte orientale de l'île de Seeland, mais aussi sur l'île plus 
petite d'Amager, laquelle se trouve face au détroit d'Øresund, qui relie la mer du Nord à la mer Baltique
et sépare le Danemark de la Suède.}

La partie ouest de \textbf{Copenhague} est relativement plate, mais on trouve un terrain plus accidenté au nord 
et au sud de la ville.
Au nord-ouest de \textbf{Copenhague} se dresse par exemple une assez grande chaîne de collines culminant à 50 m 
d'altitude. Ces paysages vallonnés sont coupés par des lacs et la rivière Mølleåen.
Le point culminant du Grand-\textbf{Copenhague} se trouve dans le bois de Rude à 91 m d'altitude. À cause de 
l'altitude plus élevée autour de Gladsaxe, on y a placé une réserve pour l'approvisionnement en eau de 
\textbf{Copenhague} ainsi que le poste émetteur de Gladsaxe.
Au sud-ouest, se dresse une déformation calcaire dans la faille de Carlsberg.
Les parties centrales de \textbf{Copenhague} se situent plutôt sur un terrain plus plat ou plus ou moins vallonné 
comme à Valby ou Brønshøj.
Deux systèmes de vallées suivent du nord-est au sud-ouest ces petites chaînes de collines ; dans l'une des
vallées, se situent les lacs du centre de \textbf{Copenhague}, dans l'autre se trouve le lac Damhussøen. Ces petites
vallées sont recoupées par les rivières Harestrup Å et Ladegårdsåen.
Amager et une grande partie de la vieille ville se situe sur un terrain côtier plat. Une partie de la 
vieille ville, y compris Christianshavn et Islandsbrygge, se trouve aujourd'hui sur une zone qui constituait
le fond marin il y a 500 ans.

D'un point de vue géologique, \textbf{Copenhague} se situe, comme la plus grande partie du Danemark, sur une moraine
de fond datant de la période glaciaire, qui elle-même repose sur un fond calcaire plus dur. À certains 
endroits, il n'y a que 10 m jusqu'au fond calcaire, ce qui posa d'importants problèmes lors de la 
construction des voies de métro.

\chapter{attractions de la ville}

\section{Quartiers}
%Liste simple
\begin{itemize}
 \item Christiana: "la ville libre de Christiania"
 \item Indre By : le centre-ville historique
 \item Nyhavn : le nouveau port
\end{itemize}

\section{Châteaux}
%Liste ordonnée
\begin{enumerate}
 \item Palais Amalienborg
 \item Christiansborg
 \item Château de Frederiksborg à Hillerød
 \item Château Kronborg à Helsingør
 \item Château de Rosenborg
 \item Rundetårn
 \item Kastellet
 
 \section{Autres}
 
 \begin{description}
 %Liste avec description + inclusion d'autres listes
  \item[musées] \hfill \\
  Listes des musées \begin{itemize}
          \item Musée des Beaux Arts Danois
          \item Musée d'art moderne Louisiana à Humlebæk
          \item Musée national du Danemark
          \item Ny Carlsberg Glyptotek
          \item La Collection Hirschsprung
         \end{itemize}

  \item[Réligieuses] \hfill \\
  Eglise \begin{enumerate}
          \item Cathédrale Notre-Dame
          \item Église Saint-Alban
          \item Église du Saint-Esprit
         \end{enumerate}
         \ldots
\end{description}


\end{enumerate}



\chapter{Transports}

\section{Sur la terre}

\subsection{velo}
\textbf{Copenhague} est l'une des villes les plus accueillantes pour les cyclistes et les piétons.
La ville fait un gros effort pour favoriser l'utilisation de la bicyclette. Pour cela, de 
nombreuses pistes cyclables (400 km) existent dans quasiment toute la ville, et des vélos 
publics sont disponibles gratuitement de mai à octobre. Chaque jour, 1,2 million de kilomètres 
sont parcourus à vélo à \textbf{Copenhague}5. 36 % des habitants de la ville vont au travail, à l'université
ou à l'école à vélo. L'objectif des autorités est de dépasser 40 % en 2012 et 50 % en 20156.
Depuis 2010, aux heures de pointe, les feux des principaux axes sont réglés sur la vitesse des 
cyclistes (vingt kilomètres à l'heure).
La ville a aussi accueilli, du 19 au 25 septembre 2011, les championnats du monde sur route.
\subsection{Transports en communs}
Il existe cinq types de transports à \textbf{Copenhague} et dans son aire urbaine : 
"le RER" (chemin de fer régional de l'île Sjælland orientale) et le chemin de fer urbain (S-tog) 
avec 6 lignes et 84 stations (170 km), le métro (2 lignes et 22 stations), les bus et le chemin de 
fer local (lokalbaner). Le réseau de S-tog est la base du maillage des transports publics de
\textbf{Copenhague}, couvrant l'essentiel de la métropole. Le métro a été mis en service en 2002, 
mais sa construction récente explique sa faible étendue. La ville compte 251 lignes de bus, qui s'étendent 
sur 4 500 km7. L'aire urbaine est un système très compliqué découpé en 95 zones. Il est possible d'effectuer 
des changements avec le même ticket, y compris d'un moyen de transport à un autre, dans une limite de durée 
qui dépend du nombre de zones pour lequel le ticket est valable.

\section{autres}
\subsection{port}
\textbf{Copenhague} n'est plus le port industriel qu'il était auparavant, mais l'activité touristique a 
pris le relais. En 2001, le port de \textbf{Copenhague} a fusionné avec celui de Malmö pour créer le 
Copenhagen-Malmö Port. L'objectif principal de cette entité est d'être le point de départ ou une escale 
des navires de croisière qui sillonnent la mer Baltique. Ainsi, en 2008, 310 bateaux de croisière y sont 
passés, pour un total de 560 000 passagers8, ce qui fait de \textbf{Copenhague} la première destination de
croisière en Europe9.
\subsection{Aéroport}
L'Aéroport de \textbf{Copenhague} (Code AITA : CPH, Code OACI : EKCH) est situé sur l'Île d'Amager, 
au Sud-Est de la Ville. C'est un aéroport international.

\part{Sciences}

\section{Codes}
\subsection{Un peu de codes}

%\lset{language=bash,numbers=left, }
\begin{framed}
\lstinputlisting{Makefile.PL}
\end{framed}
\section{mathématiques}



\end{document}          
